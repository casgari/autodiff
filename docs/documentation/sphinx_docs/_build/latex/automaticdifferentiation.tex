%% Generated by Sphinx.
\def\sphinxdocclass{report}
\documentclass[letterpaper,10pt,english]{sphinxmanual}
\ifdefined\pdfpxdimen
   \let\sphinxpxdimen\pdfpxdimen\else\newdimen\sphinxpxdimen
\fi \sphinxpxdimen=.75bp\relax
\ifdefined\pdfimageresolution
    \pdfimageresolution= \numexpr \dimexpr1in\relax/\sphinxpxdimen\relax
\fi
%% let collapsible pdf bookmarks panel have high depth per default
\PassOptionsToPackage{bookmarksdepth=5}{hyperref}


\PassOptionsToPackage{warn}{textcomp}
\usepackage[utf8]{inputenc}
\ifdefined\DeclareUnicodeCharacter
% support both utf8 and utf8x syntaxes
  \ifdefined\DeclareUnicodeCharacterAsOptional
    \def\sphinxDUC#1{\DeclareUnicodeCharacter{"#1}}
  \else
    \let\sphinxDUC\DeclareUnicodeCharacter
  \fi
  \sphinxDUC{00A0}{\nobreakspace}
  \sphinxDUC{2500}{\sphinxunichar{2500}}
  \sphinxDUC{2502}{\sphinxunichar{2502}}
  \sphinxDUC{2514}{\sphinxunichar{2514}}
  \sphinxDUC{251C}{\sphinxunichar{251C}}
  \sphinxDUC{2572}{\textbackslash}
\fi
\usepackage{cmap}
\usepackage[T1]{fontenc}
\usepackage{amsmath,amssymb,amstext}
\usepackage{babel}



\usepackage{tgtermes}
\usepackage{tgheros}
\renewcommand{\ttdefault}{txtt}



\usepackage[Bjarne]{fncychap}
\usepackage{sphinx}

\fvset{fontsize=auto}
\usepackage{geometry}


% Include hyperref last.
\usepackage{hyperref}
% Fix anchor placement for figures with captions.
\usepackage{hypcap}% it must be loaded after hyperref.
% Set up styles of URL: it should be placed after hyperref.
\urlstyle{same}

\addto\captionsenglish{\renewcommand{\contentsname}{Contents:}}

\usepackage{sphinxmessages}
\setcounter{tocdepth}{1}



\title{AutomaticDifferentiation}
\date{Dec 10, 2022}
\release{}
\author{Cyrus Asgari, Caleb Saul, Sal Blanco, James Bardin}
\newcommand{\sphinxlogo}{\vbox{}}
\renewcommand{\releasename}{}
\makeindex
\begin{document}

\ifdefined\shorthandoff
  \ifnum\catcode`\=\string=\active\shorthandoff{=}\fi
  \ifnum\catcode`\"=\active\shorthandoff{"}\fi
\fi

\pagestyle{empty}
\sphinxmaketitle
\pagestyle{plain}
\sphinxtableofcontents
\pagestyle{normal}
\phantomsection\label{\detokenize{index::doc}}


\sphinxstepscope


\chapter{src}
\label{\detokenize{modules:src}}\label{\detokenize{modules::doc}}
\sphinxstepscope


\section{autodiff package}
\label{\detokenize{autodiff:autodiff-package}}\label{\detokenize{autodiff::doc}}

\subsection{Submodules}
\label{\detokenize{autodiff:submodules}}

\subsection{autodiff.dualnumber module}
\label{\detokenize{autodiff:module-autodiff.dualnumber}}\label{\detokenize{autodiff:autodiff-dualnumber-module}}\index{module@\spxentry{module}!autodiff.dualnumber@\spxentry{autodiff.dualnumber}}\index{autodiff.dualnumber@\spxentry{autodiff.dualnumber}!module@\spxentry{module}}\index{DualNumber (class in autodiff.dualnumber)@\spxentry{DualNumber}\spxextra{class in autodiff.dualnumber}}

\begin{fulllineitems}
\phantomsection\label{\detokenize{autodiff:autodiff.dualnumber.DualNumber}}
\pysigstartsignatures
\pysiglinewithargsret{\sphinxbfcode{\sphinxupquote{class\DUrole{w}{  }}}\sphinxcode{\sphinxupquote{autodiff.dualnumber.}}\sphinxbfcode{\sphinxupquote{DualNumber}}}{\emph{\DUrole{n}{real}}, \emph{\DUrole{n}{dual}\DUrole{o}{=}\DUrole{default_value}{1}}}{}
\pysigstopsignatures
\sphinxAtStartPar
Bases: \sphinxcode{\sphinxupquote{object}}

\sphinxAtStartPar
DualNumber class with overloaded operators for forward mode AD.

\end{fulllineitems}



\subsection{autodiff.func module}
\label{\detokenize{autodiff:module-autodiff.func}}\label{\detokenize{autodiff:autodiff-func-module}}\index{module@\spxentry{module}!autodiff.func@\spxentry{autodiff.func}}\index{autodiff.func@\spxentry{autodiff.func}!module@\spxentry{module}}
\sphinxAtStartPar
Function Module with Func class definition
\index{Func (class in autodiff.func)@\spxentry{Func}\spxextra{class in autodiff.func}}

\begin{fulllineitems}
\phantomsection\label{\detokenize{autodiff:autodiff.func.Func}}
\pysigstartsignatures
\pysiglinewithargsret{\sphinxbfcode{\sphinxupquote{class\DUrole{w}{  }}}\sphinxcode{\sphinxupquote{autodiff.func.}}\sphinxbfcode{\sphinxupquote{Func}}}{\emph{\DUrole{n}{function}}, \emph{\DUrole{n}{num\_inputs}}, \emph{\DUrole{n}{num\_outputs}}}{}
\pysigstopsignatures
\sphinxAtStartPar
Bases: \sphinxcode{\sphinxupquote{object}}

\sphinxAtStartPar
Func class to be used for forward mode AD evaluation of derivatives.
\index{eval() (autodiff.func.Func method)@\spxentry{eval()}\spxextra{autodiff.func.Func method}}

\begin{fulllineitems}
\phantomsection\label{\detokenize{autodiff:autodiff.func.Func.eval}}
\pysigstartsignatures
\pysiglinewithargsret{\sphinxbfcode{\sphinxupquote{eval}}}{\emph{\DUrole{n}{point}}, \emph{\DUrole{n}{seed\_vector}}}{}
\pysigstopsignatures
\sphinxAtStartPar
Evaluate the function value and derivative in a given direction at a given point.


\subsubsection{Parameters}
\label{\detokenize{autodiff:parameters}}\begin{description}
\sphinxlineitem{point}{[}int, float, list, np.ndarray{]}
\sphinxAtStartPar
The input point at wish users wish to evaluate the function and the function’s derivative. Can be int/float if \& only if input dimension = 1.

\sphinxlineitem{seed\_vector}{[}list, np.ndarray{]}
\sphinxAtStartPar
The direction in wish users wish to evaluate the function’s derivative

\end{description}


\subsubsection{Returns}
\label{\detokenize{autodiff:returns}}\begin{description}
\sphinxlineitem{tuple}
\sphinxAtStartPar
The first element is an np.ndarray with num\_outputs length representing the function’s value
evaluated at the given point.  The second element is an np.ndarray with num\_outputs length
representing the value of the derivative at the given point in the given direction.

\end{description}


\subsubsection{Examples}
\label{\detokenize{autodiff:examples}}
\begin{sphinxVerbatim}[commandchars=\\\{\}]
\PYG{g+gp}{\PYGZgt{}\PYGZgt{}\PYGZgt{} }\PYG{n}{fofx} \PYG{o}{=} \PYG{k}{lambda} \PYG{n}{x} \PYG{p}{:} \PYG{l+m+mi}{2} \PYG{o}{*} \PYG{n}{x}\PYG{o}{*}\PYG{o}{*}\PYG{l+m+mi}{2} \PYG{o}{+} \PYG{l+m+mi}{4}
\PYG{g+gp}{\PYGZgt{}\PYGZgt{}\PYGZgt{} }\PYG{n}{f} \PYG{o}{=} \PYG{n}{Func}\PYG{p}{(}\PYG{n}{fofx}\PYG{p}{,} \PYG{l+m+mi}{1}\PYG{p}{,} \PYG{l+m+mi}{1}\PYG{p}{)}
\PYG{g+gp}{\PYGZgt{}\PYGZgt{}\PYGZgt{} }\PYG{n}{p} \PYG{o}{=} \PYG{p}{[}\PYG{l+m+mi}{4}\PYG{p}{]}
\PYG{g+gp}{\PYGZgt{}\PYGZgt{}\PYGZgt{} }\PYG{n}{direction} \PYG{o}{=} \PYG{p}{[}\PYG{l+m+mi}{1}\PYG{p}{]}
\PYG{g+gp}{\PYGZgt{}\PYGZgt{}\PYGZgt{} }\PYG{n}{f}\PYG{o}{.}\PYG{n}{eval}\PYG{p}{(}\PYG{n}{p}\PYG{p}{,} \PYG{n}{direction}\PYG{p}{)}
\PYG{g+go}{(array([36.]), array([16.]))}
\PYG{g+gp}{\PYGZgt{}\PYGZgt{}\PYGZgt{} }\PYG{k}{def} \PYG{n+nf}{fofx}\PYG{p}{(}\PYG{n}{x}\PYG{p}{,} \PYG{n}{y}\PYG{p}{)}\PYG{p}{:}
\PYG{g+go}{        return y + x**2, x \PYGZhy{} 3*y}
\PYG{g+gp}{\PYGZgt{}\PYGZgt{}\PYGZgt{} }\PYG{n}{f} \PYG{o}{=} \PYG{n}{Func}\PYG{p}{(}\PYG{n}{fofx}\PYG{p}{,}\PYG{l+m+mi}{2}\PYG{p}{,}\PYG{l+m+mi}{2}\PYG{p}{)}
\PYG{g+gp}{\PYGZgt{}\PYGZgt{}\PYGZgt{} }\PYG{n}{p} \PYG{o}{=} \PYG{p}{[}\PYG{l+m+mi}{4}\PYG{p}{,}\PYG{l+m+mi}{2}\PYG{p}{]}
\PYG{g+gp}{\PYGZgt{}\PYGZgt{}\PYGZgt{} }\PYG{n}{direction} \PYG{o}{=} \PYG{p}{[}\PYG{l+m+mi}{1}\PYG{p}{,}\PYG{l+m+mi}{0}\PYG{p}{]}
\PYG{g+gp}{\PYGZgt{}\PYGZgt{}\PYGZgt{} }\PYG{n}{f}\PYG{o}{.}\PYG{n}{eval}\PYG{p}{(}\PYG{n}{p}\PYG{p}{,} \PYG{n}{direction}\PYG{p}{)}
\PYG{g+go}{(array([18., \PYGZhy{}2.]), array([8., 1.]))}
\end{sphinxVerbatim}

\end{fulllineitems}

\index{graph() (autodiff.func.Func method)@\spxentry{graph()}\spxextra{autodiff.func.Func method}}

\begin{fulllineitems}
\phantomsection\label{\detokenize{autodiff:autodiff.func.Func.graph}}
\pysigstartsignatures
\pysiglinewithargsret{\sphinxbfcode{\sphinxupquote{graph}}}{\emph{\DUrole{n}{point}}, \emph{\DUrole{n}{graph\_direction}\DUrole{o}{=}\DUrole{default_value}{\textquotesingle{}LR\textquotesingle{}}}}{}
\pysigstopsignatures
\sphinxAtStartPar
Display the computational graph of the function


\subsubsection{Parameters}
\label{\detokenize{autodiff:id1}}\begin{description}
\sphinxlineitem{point}{[}int,float,list, np.ndarray{]}
\sphinxAtStartPar
The input point at wish users wish to calculate the function’s computational graph.

\sphinxlineitem{graph\_direction}{[}str{]}
\sphinxAtStartPar
A string indicating if users would like the computational graph display ‘LR’, left\sphinxhyphen{}to\sphinxhyphen{}right,
or ‘TB’ top\sphinxhyphen{}to\sphinxhyphen{}bottom. Defaults to ‘LR’, left\sphinxhyphen{}to\sphinxhyphen{}right

\end{description}


\subsubsection{Returns}
\label{\detokenize{autodiff:id2}}\begin{description}
\sphinxlineitem{graphviz.graphs.Digraph}
\sphinxAtStartPar
A graphviz.graphs.Digraph object that displays the computational graph of the function

\end{description}


\subsubsection{Examples}
\label{\detokenize{autodiff:id3}}
\begin{sphinxVerbatim}[commandchars=\\\{\}]
\PYG{g+gp}{\PYGZgt{}\PYGZgt{}\PYGZgt{} }\PYG{k}{def} \PYG{n+nf}{fofx}\PYG{p}{(}\PYG{n}{x}\PYG{p}{,}\PYG{n}{y}\PYG{p}{,}\PYG{n}{w}\PYG{p}{)}\PYG{p}{:}
\PYG{g+go}{    return  w + x**(2\PYGZhy{}y+w) + 2*y}
\PYG{g+gp}{\PYGZgt{}\PYGZgt{}\PYGZgt{} }\PYG{n}{f} \PYG{o}{=} \PYG{n}{Func}\PYG{p}{(}\PYG{n}{fofx}\PYG{p}{,}\PYG{l+m+mi}{3}\PYG{p}{,}\PYG{l+m+mi}{1}\PYG{p}{)}
\PYG{g+gp}{\PYGZgt{}\PYGZgt{}\PYGZgt{} }\PYG{n}{p} \PYG{o}{=} \PYG{p}{[}\PYG{l+m+mi}{5}\PYG{p}{,}\PYG{l+m+mf}{2.09}\PYG{p}{,}\PYG{l+m+mf}{4.8}\PYG{p}{]}
\PYG{g+gp}{\PYGZgt{}\PYGZgt{}\PYGZgt{} }\PYG{n}{f}\PYG{o}{.}\PYG{n}{graph}\PYG{p}{(}\PYG{n}{p}\PYG{p}{)}
\PYG{g+go}{\PYGZob{}computational graph displayed from left to right\PYGZcb{}}
\PYG{g+gp}{\PYGZgt{}\PYGZgt{}\PYGZgt{} }\PYG{n}{f}\PYG{o}{.}\PYG{n}{graph}\PYG{p}{(}\PYG{n}{p}\PYG{p}{,}\PYG{l+s+s1}{\PYGZsq{}}\PYG{l+s+s1}{TB}\PYG{l+s+s1}{\PYGZsq{}}\PYG{p}{)}
\PYG{g+go}{\PYGZob{}computational graph displayed from top to bottom\PYGZcb{}}
\end{sphinxVerbatim}

\end{fulllineitems}

\index{jacobian() (autodiff.func.Func method)@\spxentry{jacobian()}\spxextra{autodiff.func.Func method}}

\begin{fulllineitems}
\phantomsection\label{\detokenize{autodiff:autodiff.func.Func.jacobian}}
\pysigstartsignatures
\pysiglinewithargsret{\sphinxbfcode{\sphinxupquote{jacobian}}}{\emph{\DUrole{n}{point}}, \emph{\DUrole{n}{reverse}\DUrole{o}{=}\DUrole{default_value}{False}}}{}
\pysigstopsignatures
\sphinxAtStartPar
The jacobian matrix of a function at a given point.


\subsubsection{Parameters}
\label{\detokenize{autodiff:id4}}\begin{description}
\sphinxlineitem{point}{[}int,float,list, np.ndarray{]}
\sphinxAtStartPar
The input point at wish users wish to evaluate the function’s jacobian matrix. Can be int/float if \& only if input dimension = 1.

\sphinxlineitem{reverse}{[}bool{]}
\sphinxAtStartPar
Set to true if users wish to evaluate the function’s jacobian using reverse mode automatic.  Default is False.

\end{description}


\subsubsection{Returns}
\label{\detokenize{autodiff:id5}}\begin{description}
\sphinxlineitem{When the jacobian is a scalar:}\begin{description}
\sphinxlineitem{int, float}
\sphinxAtStartPar
The scalar value representing the jacobian of the function at the inputted point

\end{description}

\sphinxlineitem{Otherwise:}\begin{description}
\sphinxlineitem{np.ndarray}
\sphinxAtStartPar
The num\_outputs by num\_inputs jacobian matrix for the function at the inputted point

\end{description}

\end{description}


\subsubsection{Examples}
\label{\detokenize{autodiff:id6}}
\begin{sphinxVerbatim}[commandchars=\\\{\}]
\PYG{g+gp}{\PYGZgt{}\PYGZgt{}\PYGZgt{} }\PYG{n}{fofx} \PYG{o}{=} \PYG{k}{lambda} \PYG{n}{x} \PYG{p}{:} \PYG{l+m+mi}{4} \PYG{o}{*} \PYG{n}{x}\PYG{o}{*}\PYG{o}{*}\PYG{l+m+mi}{3} \PYG{o}{\PYGZhy{}} \PYG{l+m+mi}{3}
\PYG{g+gp}{\PYGZgt{}\PYGZgt{}\PYGZgt{} }\PYG{n}{f} \PYG{o}{=} \PYG{n}{Func}\PYG{p}{(}\PYG{n}{fofx}\PYG{p}{,} \PYG{l+m+mi}{1}\PYG{p}{,} \PYG{l+m+mi}{1}\PYG{p}{)}
\PYG{g+gp}{\PYGZgt{}\PYGZgt{}\PYGZgt{} }\PYG{n}{p} \PYG{o}{=} \PYG{p}{[}\PYG{l+m+mi}{2}\PYG{p}{]}
\PYG{g+gp}{\PYGZgt{}\PYGZgt{}\PYGZgt{} }\PYG{n}{f}\PYG{o}{.}\PYG{n}{jacobian}\PYG{p}{(}\PYG{n}{p}\PYG{p}{)}
\PYG{g+go}{48.0}
\PYG{g+gp}{\PYGZgt{}\PYGZgt{}\PYGZgt{} }\PYG{k}{def} \PYG{n+nf}{fofx}\PYG{p}{(}\PYG{n}{x}\PYG{p}{,}\PYG{n}{y}\PYG{p}{)}\PYG{p}{:}
\PYG{g+go}{        return y * x**2, (x\PYGZhy{}y)}
\PYG{g+gp}{\PYGZgt{}\PYGZgt{}\PYGZgt{} }\PYG{n}{f} \PYG{o}{=} \PYG{n}{Func}\PYG{p}{(}\PYG{n}{fofx}\PYG{p}{,}\PYG{l+m+mi}{2}\PYG{p}{,}\PYG{l+m+mi}{2}\PYG{p}{)}
\PYG{g+gp}{\PYGZgt{}\PYGZgt{}\PYGZgt{} }\PYG{n}{p} \PYG{o}{=} \PYG{p}{[}\PYG{l+m+mi}{2}\PYG{p}{,}\PYG{l+m+mi}{3}\PYG{p}{]}
\PYG{g+gp}{\PYGZgt{}\PYGZgt{}\PYGZgt{} }\PYG{n}{f}\PYG{o}{.}\PYG{n}{jacobian}\PYG{p}{(}\PYG{n}{p}\PYG{p}{)}
\PYG{g+go}{array([[12.,  4.],}
\PYG{g+go}{      [ 1., \PYGZhy{}1.]])}
\end{sphinxVerbatim}

\end{fulllineitems}


\end{fulllineitems}



\subsection{autodiff.operators module}
\label{\detokenize{autodiff:module-autodiff.operators}}\label{\detokenize{autodiff:autodiff-operators-module}}\index{module@\spxentry{module}!autodiff.operators@\spxentry{autodiff.operators}}\index{autodiff.operators@\spxentry{autodiff.operators}!module@\spxentry{module}}\index{arccos() (in module autodiff.operators)@\spxentry{arccos()}\spxextra{in module autodiff.operators}}

\begin{fulllineitems}
\phantomsection\label{\detokenize{autodiff:autodiff.operators.arccos}}
\pysigstartsignatures
\pysiglinewithargsret{\sphinxcode{\sphinxupquote{autodiff.operators.}}\sphinxbfcode{\sphinxupquote{arccos}}}{\emph{\DUrole{n}{x}}}{}
\pysigstopsignatures
\sphinxAtStartPar
Arccos operation for dual numbers and ints or floats


\subsubsection{Parameters}
\label{\detokenize{autodiff:id7}}\begin{description}
\sphinxlineitem{x}{[}DualNumber, int, float{]}
\sphinxAtStartPar
The DualNumber, int, or float used for the arccos operation

\end{description}


\subsubsection{Returns}
\label{\detokenize{autodiff:id8}}\begin{description}
\sphinxlineitem{DualNumber}
\sphinxAtStartPar
Returns a new DualNumber object with the arccos operation performed. Integer and
floats are supported, along with the arccos of dual numbers.

\end{description}


\subsubsection{Raises}
\label{\detokenize{autodiff:raises}}
\sphinxAtStartPar
TypeError : raise TypeError if argument is not of int, float, or DualNumber type
ValueError : raise ValueError if argument is not between \sphinxhyphen{}1 and 1 inclusive

\end{fulllineitems}

\index{arcsin() (in module autodiff.operators)@\spxentry{arcsin()}\spxextra{in module autodiff.operators}}

\begin{fulllineitems}
\phantomsection\label{\detokenize{autodiff:autodiff.operators.arcsin}}
\pysigstartsignatures
\pysiglinewithargsret{\sphinxcode{\sphinxupquote{autodiff.operators.}}\sphinxbfcode{\sphinxupquote{arcsin}}}{\emph{\DUrole{n}{x}}}{}
\pysigstopsignatures
\sphinxAtStartPar
Arcsin operation for dual numbers and ints or floats


\subsubsection{Parameters}
\label{\detokenize{autodiff:id9}}\begin{description}
\sphinxlineitem{x}{[}DualNumber, int, float{]}
\sphinxAtStartPar
The DualNumber, int, or float used for the arcsin operation

\end{description}


\subsubsection{Returns}
\label{\detokenize{autodiff:id10}}\begin{description}
\sphinxlineitem{DualNumber}
\sphinxAtStartPar
Returns a new DualNumber object with the arcsin operation performed. Integer and
floats are supported, along with the arcsin of dual numbers.

\end{description}


\subsubsection{Raises}
\label{\detokenize{autodiff:id11}}
\sphinxAtStartPar
TypeError : raise TypeError if argument is not of int, float, or DualNumber type
ValueError : raise ValueError if argument is not between \sphinxhyphen{}1 and 1 inclusive

\end{fulllineitems}

\index{arctan() (in module autodiff.operators)@\spxentry{arctan()}\spxextra{in module autodiff.operators}}

\begin{fulllineitems}
\phantomsection\label{\detokenize{autodiff:autodiff.operators.arctan}}
\pysigstartsignatures
\pysiglinewithargsret{\sphinxcode{\sphinxupquote{autodiff.operators.}}\sphinxbfcode{\sphinxupquote{arctan}}}{\emph{\DUrole{n}{x}}}{}
\pysigstopsignatures
\sphinxAtStartPar
Arctan operation for dual numbers and ints or floats


\subsubsection{Parameters}
\label{\detokenize{autodiff:id12}}\begin{description}
\sphinxlineitem{x}{[}DualNumber, int, float{]}
\sphinxAtStartPar
The DualNumber, int, or float used for the arctan operation

\end{description}


\subsubsection{Returns}
\label{\detokenize{autodiff:id13}}\begin{description}
\sphinxlineitem{DualNumber}
\sphinxAtStartPar
Returns a new DualNumber object with the arctan operation performed. Integer and
floats are supported, along with the arctan of dual numbers.

\end{description}


\subsubsection{Raises}
\label{\detokenize{autodiff:id14}}
\sphinxAtStartPar
TypeError : raise TypeError if argument is not of int, float, or DualNumber type

\end{fulllineitems}

\index{cos() (in module autodiff.operators)@\spxentry{cos()}\spxextra{in module autodiff.operators}}

\begin{fulllineitems}
\phantomsection\label{\detokenize{autodiff:autodiff.operators.cos}}
\pysigstartsignatures
\pysiglinewithargsret{\sphinxcode{\sphinxupquote{autodiff.operators.}}\sphinxbfcode{\sphinxupquote{cos}}}{\emph{\DUrole{n}{x}}}{}
\pysigstopsignatures
\sphinxAtStartPar
Cosine operation for dual numbers and ints or floats


\subsubsection{Parameters}
\label{\detokenize{autodiff:id15}}\begin{description}
\sphinxlineitem{x}{[}DualNumber, int, float{]}
\sphinxAtStartPar
The DualNumber, int, or float the cosine is evaluated on

\end{description}


\subsubsection{Returns}
\label{\detokenize{autodiff:id16}}\begin{description}
\sphinxlineitem{DualNumber}
\sphinxAtStartPar
Returns a new DualNumber object with the cosine operation performed. Integer and
floats are supported, along with the cosine of dual numbers through application 
of the chain rule.

\end{description}


\subsubsection{Raises}
\label{\detokenize{autodiff:id17}}
\sphinxAtStartPar
TypeError : raise TypeError if argument is not of int, float, or DualNumber type

\end{fulllineitems}

\index{cosh() (in module autodiff.operators)@\spxentry{cosh()}\spxextra{in module autodiff.operators}}

\begin{fulllineitems}
\phantomsection\label{\detokenize{autodiff:autodiff.operators.cosh}}
\pysigstartsignatures
\pysiglinewithargsret{\sphinxcode{\sphinxupquote{autodiff.operators.}}\sphinxbfcode{\sphinxupquote{cosh}}}{\emph{\DUrole{n}{x}}}{}
\pysigstopsignatures
\sphinxAtStartPar
Cosh operation for dual numbers and ints or floats


\subsubsection{Parameters}
\label{\detokenize{autodiff:id18}}\begin{description}
\sphinxlineitem{x}{[}DualNumber, int, float{]}
\sphinxAtStartPar
The DualNumber, int, or float used for the cosh operation

\end{description}


\subsubsection{Returns}
\label{\detokenize{autodiff:id19}}\begin{description}
\sphinxlineitem{DualNumber}
\sphinxAtStartPar
Returns a new DualNumber object with the cosh operation performed. Integer and
floats are supported, along with the cosh of dual numbers.

\end{description}


\subsubsection{Raises}
\label{\detokenize{autodiff:id20}}
\sphinxAtStartPar
TypeError : raise TypeError if argument is not of int, float, or DualNumber type

\end{fulllineitems}

\index{exp() (in module autodiff.operators)@\spxentry{exp()}\spxextra{in module autodiff.operators}}

\begin{fulllineitems}
\phantomsection\label{\detokenize{autodiff:autodiff.operators.exp}}
\pysigstartsignatures
\pysiglinewithargsret{\sphinxcode{\sphinxupquote{autodiff.operators.}}\sphinxbfcode{\sphinxupquote{exp}}}{\emph{\DUrole{n}{x}}}{}
\pysigstopsignatures
\sphinxAtStartPar
Exp operation for dual numbers and ints or floats (base e)


\subsubsection{Parameters}
\label{\detokenize{autodiff:id21}}\begin{description}
\sphinxlineitem{x}{[}DualNumber, int, float{]}
\sphinxAtStartPar
The DualNumber, int, or float being exponentiated

\end{description}


\subsubsection{Returns}
\label{\detokenize{autodiff:id22}}\begin{description}
\sphinxlineitem{DualNumber}
\sphinxAtStartPar
Returns a new DualNumber object with the exp operation performed. Integer and
floats are supported, along with the exponent of dual numbers.

\end{description}


\subsubsection{Raises}
\label{\detokenize{autodiff:id23}}
\sphinxAtStartPar
TypeError : raise TypeError if argument is not of int, float, or DualNumber type

\end{fulllineitems}

\index{log() (in module autodiff.operators)@\spxentry{log()}\spxextra{in module autodiff.operators}}

\begin{fulllineitems}
\phantomsection\label{\detokenize{autodiff:autodiff.operators.log}}
\pysigstartsignatures
\pysiglinewithargsret{\sphinxcode{\sphinxupquote{autodiff.operators.}}\sphinxbfcode{\sphinxupquote{log}}}{\emph{\DUrole{n}{x}}}{}
\pysigstopsignatures
\sphinxAtStartPar
Log operation for dual numbers and ints or floats (base e)


\subsubsection{Parameters}
\label{\detokenize{autodiff:id24}}\begin{description}
\sphinxlineitem{x}{[}DualNumber, int, float{]}
\sphinxAtStartPar
The DualNumber, int, or float used for the natural logarithm

\end{description}


\subsubsection{Returns}
\label{\detokenize{autodiff:id25}}\begin{description}
\sphinxlineitem{DualNumber}
\sphinxAtStartPar
Returns a new DualNumber object with the logarithm operation performed. Integer and
floats are supported, along with the natural logarithm of dual numbers.

\end{description}


\subsubsection{Raises}
\label{\detokenize{autodiff:id26}}
\sphinxAtStartPar
TypeError : raise TypeError if argument is not of int, float, or DualNumber type
ValueError : raise ValueError if argument is nonpositive

\end{fulllineitems}

\index{log\_base() (in module autodiff.operators)@\spxentry{log\_base()}\spxextra{in module autodiff.operators}}

\begin{fulllineitems}
\phantomsection\label{\detokenize{autodiff:autodiff.operators.log_base}}
\pysigstartsignatures
\pysiglinewithargsret{\sphinxcode{\sphinxupquote{autodiff.operators.}}\sphinxbfcode{\sphinxupquote{log\_base}}}{\emph{\DUrole{n}{x}}, \emph{\DUrole{n}{base}}}{}
\pysigstopsignatures
\sphinxAtStartPar
Log base operation for dual numbers and ints or floats


\subsubsection{Parameters}
\label{\detokenize{autodiff:id27}}\begin{description}
\sphinxlineitem{x}{[}DualNumber, int, float{]}
\sphinxAtStartPar
The DualNumber, int, or float argument for the logarithm

\sphinxlineitem{x}{[}int, float{]}
\sphinxAtStartPar
The int, or float base of the logarithm

\end{description}


\subsubsection{Returns}
\label{\detokenize{autodiff:id28}}\begin{description}
\sphinxlineitem{DualNumber}
\sphinxAtStartPar
Returns a new DualNumber object with the logarithm operation performed. Integer and
floats are supported, along with the logarithm of dual numbers of any int/float base.

\end{description}


\subsubsection{Raises}
\label{\detokenize{autodiff:id29}}
\sphinxAtStartPar
TypeError : raise TypeError if argument is not of int, float, or DualNumber type
ValueError : raise ValueError if argument is nonpositive

\end{fulllineitems}

\index{logistic() (in module autodiff.operators)@\spxentry{logistic()}\spxextra{in module autodiff.operators}}

\begin{fulllineitems}
\phantomsection\label{\detokenize{autodiff:autodiff.operators.logistic}}
\pysigstartsignatures
\pysiglinewithargsret{\sphinxcode{\sphinxupquote{autodiff.operators.}}\sphinxbfcode{\sphinxupquote{logistic}}}{\emph{\DUrole{n}{x}}}{}
\pysigstopsignatures
\sphinxAtStartPar
Logistic operation for dual numbers and ints or floats


\subsubsection{Parameters}
\label{\detokenize{autodiff:id30}}\begin{description}
\sphinxlineitem{x}{[}DualNumber, int, float{]}
\sphinxAtStartPar
The DualNumber, int, or float used for the tanh operation

\end{description}


\subsubsection{Returns}
\label{\detokenize{autodiff:id31}}\begin{description}
\sphinxlineitem{DualNumber}
\sphinxAtStartPar
Returns a new DualNumber object with the tanh operation performed. Integer and
floats are supported, along with the tanh of dual numbers.

\end{description}


\subsubsection{Raises}
\label{\detokenize{autodiff:id32}}
\sphinxAtStartPar
TypeError : raise TypeError if argument is not of int, float, or DualNumber type

\end{fulllineitems}

\index{sin() (in module autodiff.operators)@\spxentry{sin()}\spxextra{in module autodiff.operators}}

\begin{fulllineitems}
\phantomsection\label{\detokenize{autodiff:autodiff.operators.sin}}
\pysigstartsignatures
\pysiglinewithargsret{\sphinxcode{\sphinxupquote{autodiff.operators.}}\sphinxbfcode{\sphinxupquote{sin}}}{\emph{\DUrole{n}{x}}}{}
\pysigstopsignatures
\sphinxAtStartPar
Sine operation for dual numbers and ints or floats


\subsubsection{Parameters}
\label{\detokenize{autodiff:id33}}\begin{description}
\sphinxlineitem{x}{[}DualNumber, int, float{]}
\sphinxAtStartPar
The DualNumber, int, or float the sine is evaluated on

\end{description}


\subsubsection{Returns}
\label{\detokenize{autodiff:id34}}\begin{description}
\sphinxlineitem{DualNumber}
\sphinxAtStartPar
Returns a new DualNumber object with the sine operation performed. Integer and
floats are supported, along with the sine of dual numbers through application 
of the chain rule.

\end{description}


\subsubsection{Raises}
\label{\detokenize{autodiff:id35}}
\sphinxAtStartPar
TypeError : raise TypeError if argument is not of int, float, or DualNumber type

\end{fulllineitems}

\index{sinh() (in module autodiff.operators)@\spxentry{sinh()}\spxextra{in module autodiff.operators}}

\begin{fulllineitems}
\phantomsection\label{\detokenize{autodiff:autodiff.operators.sinh}}
\pysigstartsignatures
\pysiglinewithargsret{\sphinxcode{\sphinxupquote{autodiff.operators.}}\sphinxbfcode{\sphinxupquote{sinh}}}{\emph{\DUrole{n}{x}}}{}
\pysigstopsignatures
\sphinxAtStartPar
Sinh operation for dual numbers and ints or floats


\subsubsection{Parameters}
\label{\detokenize{autodiff:id36}}\begin{description}
\sphinxlineitem{x}{[}DualNumber, int, float{]}
\sphinxAtStartPar
The DualNumber, int, or float used for the sinh operation

\end{description}


\subsubsection{Returns}
\label{\detokenize{autodiff:id37}}\begin{description}
\sphinxlineitem{DualNumber}
\sphinxAtStartPar
Returns a new DualNumber object with the sinh operation performed. Integer and
floats are supported, along with the sinh of dual numbers.

\end{description}


\subsubsection{Raises}
\label{\detokenize{autodiff:id38}}
\sphinxAtStartPar
TypeError : raise TypeError if argument is not of int, float, or DualNumber type

\end{fulllineitems}

\index{sqrt() (in module autodiff.operators)@\spxentry{sqrt()}\spxextra{in module autodiff.operators}}

\begin{fulllineitems}
\phantomsection\label{\detokenize{autodiff:autodiff.operators.sqrt}}
\pysigstartsignatures
\pysiglinewithargsret{\sphinxcode{\sphinxupquote{autodiff.operators.}}\sphinxbfcode{\sphinxupquote{sqrt}}}{\emph{\DUrole{n}{x}}}{}
\pysigstopsignatures
\sphinxAtStartPar
Square root operation for dual numbers and ints or floats


\subsubsection{Parameters}
\label{\detokenize{autodiff:id39}}\begin{description}
\sphinxlineitem{x}{[}DualNumber, int, float{]}
\sphinxAtStartPar
The DualNumber, int, or float used for the sqrt operation

\end{description}


\subsubsection{Returns}
\label{\detokenize{autodiff:id40}}\begin{description}
\sphinxlineitem{DualNumber}
\sphinxAtStartPar
Returns a new DualNumber object with the sqrt operation performed. Integer and
floats are supported, along with the sqrt of dual numbers.

\end{description}


\subsubsection{Raises}
\label{\detokenize{autodiff:id41}}
\sphinxAtStartPar
TypeError : raise TypeError if argument is not of int, float, or DualNumber type
ValueError : raise ValueError if argument is negative

\end{fulllineitems}

\index{tan() (in module autodiff.operators)@\spxentry{tan()}\spxextra{in module autodiff.operators}}

\begin{fulllineitems}
\phantomsection\label{\detokenize{autodiff:autodiff.operators.tan}}
\pysigstartsignatures
\pysiglinewithargsret{\sphinxcode{\sphinxupquote{autodiff.operators.}}\sphinxbfcode{\sphinxupquote{tan}}}{\emph{\DUrole{n}{x}}}{}
\pysigstopsignatures
\sphinxAtStartPar
Tangent operation for dual numbers and ints or floats


\subsubsection{Parameters}
\label{\detokenize{autodiff:id42}}\begin{description}
\sphinxlineitem{x}{[}DualNumber, int, float{]}
\sphinxAtStartPar
The DualNumber, int, or float the tangent is evaluated on

\end{description}


\subsubsection{Returns}
\label{\detokenize{autodiff:id43}}\begin{description}
\sphinxlineitem{DualNumber}
\sphinxAtStartPar
Returns a new DualNumber object with the tangent operation performed. Integer and
floats are supported, along with the tangent of dual numbers through application 
of the chain rule.

\end{description}


\subsubsection{Raises}
\label{\detokenize{autodiff:id44}}
\sphinxAtStartPar
TypeError : raise TypeError if argument is not of int, float, or DualNumber type

\end{fulllineitems}

\index{tanh() (in module autodiff.operators)@\spxentry{tanh()}\spxextra{in module autodiff.operators}}

\begin{fulllineitems}
\phantomsection\label{\detokenize{autodiff:autodiff.operators.tanh}}
\pysigstartsignatures
\pysiglinewithargsret{\sphinxcode{\sphinxupquote{autodiff.operators.}}\sphinxbfcode{\sphinxupquote{tanh}}}{\emph{\DUrole{n}{x}}}{}
\pysigstopsignatures
\sphinxAtStartPar
Tanh operation for dual numbers and ints or floats


\subsubsection{Parameters}
\label{\detokenize{autodiff:id45}}\begin{description}
\sphinxlineitem{x}{[}DualNumber, int, float{]}
\sphinxAtStartPar
The DualNumber, int, or float used for the tanh operation

\end{description}


\subsubsection{Returns}
\label{\detokenize{autodiff:id46}}\begin{description}
\sphinxlineitem{DualNumber}
\sphinxAtStartPar
Returns a new DualNumber object with the tanh operation performed. Integer and
floats are supported, along with the tanh of dual numbers.

\end{description}


\subsubsection{Raises}
\label{\detokenize{autodiff:id47}}
\sphinxAtStartPar
TypeError : raise TypeError if argument is not of int, float, or DualNumber type

\end{fulllineitems}



\subsection{autodiff.rNode module}
\label{\detokenize{autodiff:module-autodiff.rNode}}\label{\detokenize{autodiff:autodiff-rnode-module}}\index{module@\spxentry{module}!autodiff.rNode@\spxentry{autodiff.rNode}}\index{autodiff.rNode@\spxentry{autodiff.rNode}!module@\spxentry{module}}\index{rNode (class in autodiff.rNode)@\spxentry{rNode}\spxextra{class in autodiff.rNode}}

\begin{fulllineitems}
\phantomsection\label{\detokenize{autodiff:autodiff.rNode.rNode}}
\pysigstartsignatures
\pysiglinewithargsret{\sphinxbfcode{\sphinxupquote{class\DUrole{w}{  }}}\sphinxcode{\sphinxupquote{autodiff.rNode.}}\sphinxbfcode{\sphinxupquote{rNode}}}{\emph{\DUrole{n}{val}}, \emph{\DUrole{n}{parents}\DUrole{o}{=}\DUrole{default_value}{()}}, \emph{\DUrole{n}{oper}\DUrole{o}{=}\DUrole{default_value}{\textquotesingle{}\textquotesingle{}}}}{}
\pysigstopsignatures
\sphinxAtStartPar
Bases: \sphinxcode{\sphinxupquote{object}}

\sphinxAtStartPar
rNode class with overloaded operators and graph compatibility for reverse mode AD.


\subsubsection{Attributes}
\label{\detokenize{autodiff:attributes}}\begin{description}
\sphinxlineitem{increment}{[}int{]}
\sphinxAtStartPar
Increment to keep track of how many instances of the rNode
have been created before the jacobian is found

\end{description}
\index{backward() (autodiff.rNode.rNode method)@\spxentry{backward()}\spxextra{autodiff.rNode.rNode method}}

\begin{fulllineitems}
\phantomsection\label{\detokenize{autodiff:autodiff.rNode.rNode.backward}}
\pysigstartsignatures
\pysiglinewithargsret{\sphinxbfcode{\sphinxupquote{backward}}}{\emph{\DUrole{n}{num\_inputs}}}{}
\pysigstopsignatures
\sphinxAtStartPar
Reverse pass to accumulative gradients


\paragraph{Parameters}
\label{\detokenize{autodiff:id48}}\begin{description}
\sphinxlineitem{num\_inputs}{[}int{]}
\sphinxAtStartPar
The number of inputs the function has

\end{description}


\paragraph{Returns}
\label{\detokenize{autodiff:id49}}\begin{description}
\sphinxlineitem{np.ndarray}
\sphinxAtStartPar
Returns a new np.ndarray with each entry corresponding to the derivative
of self node with respect one of the function inputs, such that, if this
node is a function output, the array itself corresponds to one row of
the jacobian matrix

\end{description}

\end{fulllineitems}

\index{graph\_object() (autodiff.rNode.rNode method)@\spxentry{graph\_object()}\spxextra{autodiff.rNode.rNode method}}

\begin{fulllineitems}
\phantomsection\label{\detokenize{autodiff:autodiff.rNode.rNode.graph_object}}
\pysigstartsignatures
\pysiglinewithargsret{\sphinxbfcode{\sphinxupquote{graph\_object}}}{}{}
\pysigstopsignatures
\sphinxAtStartPar
Create list of nodes and edges used to construct computational graph


\paragraph{Returns}
\label{\detokenize{autodiff:id50}}\begin{description}
\sphinxlineitem{tuple}
\sphinxAtStartPar
Returns a tuple of two sets. The first set is a set of all the nodes
in the computational graph. The second set is a set of tuples where each tuple
has rNode objects as entries, and represent rNodes that should have an edge
between them in the computational graph.

\end{description}

\end{fulllineitems}

\index{increment (autodiff.rNode.rNode attribute)@\spxentry{increment}\spxextra{autodiff.rNode.rNode attribute}}

\begin{fulllineitems}
\phantomsection\label{\detokenize{autodiff:autodiff.rNode.rNode.increment}}
\pysigstartsignatures
\pysigline{\sphinxbfcode{\sphinxupquote{increment}}\sphinxbfcode{\sphinxupquote{\DUrole{w}{  }\DUrole{p}{=}\DUrole{w}{  }0}}}
\pysigstopsignatures
\end{fulllineitems}


\end{fulllineitems}



\subsection{Module contents}
\label{\detokenize{autodiff:module-autodiff}}\label{\detokenize{autodiff:module-contents}}\index{module@\spxentry{module}!autodiff@\spxentry{autodiff}}\index{autodiff@\spxentry{autodiff}!module@\spxentry{module}}\index{DualNumber (class in autodiff)@\spxentry{DualNumber}\spxextra{class in autodiff}}

\begin{fulllineitems}
\phantomsection\label{\detokenize{autodiff:autodiff.DualNumber}}
\pysigstartsignatures
\pysiglinewithargsret{\sphinxbfcode{\sphinxupquote{class\DUrole{w}{  }}}\sphinxcode{\sphinxupquote{autodiff.}}\sphinxbfcode{\sphinxupquote{DualNumber}}}{\emph{\DUrole{n}{real}}, \emph{\DUrole{n}{dual}\DUrole{o}{=}\DUrole{default_value}{1}}}{}
\pysigstopsignatures
\sphinxAtStartPar
Bases: \sphinxcode{\sphinxupquote{object}}

\sphinxAtStartPar
DualNumber class with overloaded operators for forward mode AD.

\end{fulllineitems}

\index{cos() (in module autodiff)@\spxentry{cos()}\spxextra{in module autodiff}}

\begin{fulllineitems}
\phantomsection\label{\detokenize{autodiff:autodiff.cos}}
\pysigstartsignatures
\pysiglinewithargsret{\sphinxcode{\sphinxupquote{autodiff.}}\sphinxbfcode{\sphinxupquote{cos}}}{\emph{\DUrole{n}{x}}}{}
\pysigstopsignatures
\sphinxAtStartPar
Cosine operation for dual numbers and ints or floats


\subsubsection{Parameters}
\label{\detokenize{autodiff:id51}}\begin{description}
\sphinxlineitem{x}{[}DualNumber, int, float{]}
\sphinxAtStartPar
The DualNumber, int, or float the cosine is evaluated on

\end{description}


\subsubsection{Returns}
\label{\detokenize{autodiff:id52}}\begin{description}
\sphinxlineitem{DualNumber}
\sphinxAtStartPar
Returns a new DualNumber object with the cosine operation performed. Integer and
floats are supported, along with the cosine of dual numbers through application 
of the chain rule.

\end{description}


\subsubsection{Raises}
\label{\detokenize{autodiff:id53}}
\sphinxAtStartPar
TypeError : raise TypeError if argument is not of int, float, or DualNumber type

\end{fulllineitems}

\index{exp() (in module autodiff)@\spxentry{exp()}\spxextra{in module autodiff}}

\begin{fulllineitems}
\phantomsection\label{\detokenize{autodiff:autodiff.exp}}
\pysigstartsignatures
\pysiglinewithargsret{\sphinxcode{\sphinxupquote{autodiff.}}\sphinxbfcode{\sphinxupquote{exp}}}{\emph{\DUrole{n}{x}}}{}
\pysigstopsignatures
\sphinxAtStartPar
Exp operation for dual numbers and ints or floats (base e)


\subsubsection{Parameters}
\label{\detokenize{autodiff:id54}}\begin{description}
\sphinxlineitem{x}{[}DualNumber, int, float{]}
\sphinxAtStartPar
The DualNumber, int, or float being exponentiated

\end{description}


\subsubsection{Returns}
\label{\detokenize{autodiff:id55}}\begin{description}
\sphinxlineitem{DualNumber}
\sphinxAtStartPar
Returns a new DualNumber object with the exp operation performed. Integer and
floats are supported, along with the exponent of dual numbers.

\end{description}


\subsubsection{Raises}
\label{\detokenize{autodiff:id56}}
\sphinxAtStartPar
TypeError : raise TypeError if argument is not of int, float, or DualNumber type

\end{fulllineitems}

\index{log() (in module autodiff)@\spxentry{log()}\spxextra{in module autodiff}}

\begin{fulllineitems}
\phantomsection\label{\detokenize{autodiff:autodiff.log}}
\pysigstartsignatures
\pysiglinewithargsret{\sphinxcode{\sphinxupquote{autodiff.}}\sphinxbfcode{\sphinxupquote{log}}}{\emph{\DUrole{n}{x}}}{}
\pysigstopsignatures
\sphinxAtStartPar
Log operation for dual numbers and ints or floats (base e)


\subsubsection{Parameters}
\label{\detokenize{autodiff:id57}}\begin{description}
\sphinxlineitem{x}{[}DualNumber, int, float{]}
\sphinxAtStartPar
The DualNumber, int, or float used for the natural logarithm

\end{description}


\subsubsection{Returns}
\label{\detokenize{autodiff:id58}}\begin{description}
\sphinxlineitem{DualNumber}
\sphinxAtStartPar
Returns a new DualNumber object with the logarithm operation performed. Integer and
floats are supported, along with the natural logarithm of dual numbers.

\end{description}


\subsubsection{Raises}
\label{\detokenize{autodiff:id59}}
\sphinxAtStartPar
TypeError : raise TypeError if argument is not of int, float, or DualNumber type
ValueError : raise ValueError if argument is nonpositive

\end{fulllineitems}

\index{sin() (in module autodiff)@\spxentry{sin()}\spxextra{in module autodiff}}

\begin{fulllineitems}
\phantomsection\label{\detokenize{autodiff:autodiff.sin}}
\pysigstartsignatures
\pysiglinewithargsret{\sphinxcode{\sphinxupquote{autodiff.}}\sphinxbfcode{\sphinxupquote{sin}}}{\emph{\DUrole{n}{x}}}{}
\pysigstopsignatures
\sphinxAtStartPar
Sine operation for dual numbers and ints or floats


\subsubsection{Parameters}
\label{\detokenize{autodiff:id60}}\begin{description}
\sphinxlineitem{x}{[}DualNumber, int, float{]}
\sphinxAtStartPar
The DualNumber, int, or float the sine is evaluated on

\end{description}


\subsubsection{Returns}
\label{\detokenize{autodiff:id61}}\begin{description}
\sphinxlineitem{DualNumber}
\sphinxAtStartPar
Returns a new DualNumber object with the sine operation performed. Integer and
floats are supported, along with the sine of dual numbers through application 
of the chain rule.

\end{description}


\subsubsection{Raises}
\label{\detokenize{autodiff:id62}}
\sphinxAtStartPar
TypeError : raise TypeError if argument is not of int, float, or DualNumber type

\end{fulllineitems}

\index{tan() (in module autodiff)@\spxentry{tan()}\spxextra{in module autodiff}}

\begin{fulllineitems}
\phantomsection\label{\detokenize{autodiff:autodiff.tan}}
\pysigstartsignatures
\pysiglinewithargsret{\sphinxcode{\sphinxupquote{autodiff.}}\sphinxbfcode{\sphinxupquote{tan}}}{\emph{\DUrole{n}{x}}}{}
\pysigstopsignatures
\sphinxAtStartPar
Tangent operation for dual numbers and ints or floats


\subsubsection{Parameters}
\label{\detokenize{autodiff:id63}}\begin{description}
\sphinxlineitem{x}{[}DualNumber, int, float{]}
\sphinxAtStartPar
The DualNumber, int, or float the tangent is evaluated on

\end{description}


\subsubsection{Returns}
\label{\detokenize{autodiff:id64}}\begin{description}
\sphinxlineitem{DualNumber}
\sphinxAtStartPar
Returns a new DualNumber object with the tangent operation performed. Integer and
floats are supported, along with the tangent of dual numbers through application 
of the chain rule.

\end{description}


\subsubsection{Raises}
\label{\detokenize{autodiff:id65}}
\sphinxAtStartPar
TypeError : raise TypeError if argument is not of int, float, or DualNumber type

\end{fulllineitems}



\chapter{Indices and tables}
\label{\detokenize{index:indices-and-tables}}\begin{itemize}
\item {} 
\sphinxAtStartPar
\DUrole{xref,std,std-ref}{genindex}

\item {} 
\sphinxAtStartPar
\DUrole{xref,std,std-ref}{modindex}

\item {} 
\sphinxAtStartPar
\DUrole{xref,std,std-ref}{search}

\end{itemize}


\renewcommand{\indexname}{Python Module Index}
\begin{sphinxtheindex}
\let\bigletter\sphinxstyleindexlettergroup
\bigletter{a}
\item\relax\sphinxstyleindexentry{autodiff}\sphinxstyleindexpageref{autodiff:\detokenize{module-autodiff}}
\item\relax\sphinxstyleindexentry{autodiff.dualnumber}\sphinxstyleindexpageref{autodiff:\detokenize{module-autodiff.dualnumber}}
\item\relax\sphinxstyleindexentry{autodiff.func}\sphinxstyleindexpageref{autodiff:\detokenize{module-autodiff.func}}
\item\relax\sphinxstyleindexentry{autodiff.operators}\sphinxstyleindexpageref{autodiff:\detokenize{module-autodiff.operators}}
\item\relax\sphinxstyleindexentry{autodiff.rNode}\sphinxstyleindexpageref{autodiff:\detokenize{module-autodiff.rNode}}
\end{sphinxtheindex}

\renewcommand{\indexname}{Index}
\printindex
\end{document}